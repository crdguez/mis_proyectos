% Template created by Karol Kozioł (www.karol-koziol.net) for ShareLaTeX

%\documentclass[a4paper,spanish,9pt]{extarticle}
%\usepackage[utf8]{inputenc}
%
%\usepackage[T1]{fontenc}
%\usepackage{verbatim}
%\usepackage{graphicx}
%\usepackage{xcolor}
%\usepackage{pgf,tikz}
%\usepackage{mathrsfs}
%
%\usetikzlibrary{shapes, calc, shapes, arrows, math, babel}
%
%\usepackage{amsmath,amssymb,textcomp}
%\everymath{\displaystyle}
%
%\usepackage{times}
%\renewcommand\familydefault{\sfdefault}
%\usepackage{tgheros}
%\usepackage[defaultmono,scale=0.85]{droidmono}
%
%\usepackage{multicol}
%\setlength{\columnseprule}{0pt}
%\setlength{\columnsep}{20.0pt}
%
%\usepackage[utf8]{inputenc}
%\usepackage[spanish]{babel}
%\usepackage{eurosym}
%
%\usepackage{graphicx}
%\graphicspath{{../img/}}
%\usepackage{svg}
%
%\usepackage{hyperref}
%
%\usepackage{geometry}
%\geometry{
%a4paper,
%total={210mm,297mm},
%left=10mm,right=10mm,top=10mm,bottom=15mm}
%
%\linespread{1.3}
%
%\newcommand{\samedir}{\mathbin{\!/\mkern-5mu/\!}}
%
%% custom title
%\makeatletter
%\renewcommand*{\maketitle}{%
%\noindent
%\begin{minipage}{0.6\textwidth}
%\begin{tikzpicture}
%\node[rectangle,rounded corners=6pt,inner sep=10pt,fill=blue!50!black,text width= 0.95\textwidth] {\color{white}\Huge \@title};
%\end{tikzpicture}
%\end{minipage}
%\hfill
%\begin{minipage}{0.35\textwidth}
%\begin{tikzpicture}
%\node[rectangle,rounded corners=3pt,inner sep=10pt,draw=blue!50!black,text width= 0.95\textwidth] {\begin{tabular}{cc} \multirow{2}{1cm}{\includegraphics[width=0.15\columnwidth]{header_right}}& \@author \\ & \ies \end{tabular}};
%\end{tikzpicture}
%\end{minipage}
%\bigskip\bigskip
%}%
%\makeatother
%
%% custom section
%\usepackage[explicit]{titlesec}
%\newcommand*\sectionlabel{}
%\titleformat{\section}
%  {\gdef\sectionlabel{}
%   \normalfont\sffamily\Large\bfseries\scshape}
%  {\gdef\sectionlabel{\thesection\ }}{0pt}
%  {
%\noindent
%\begin{tikzpicture}
%\node[rectangle,rounded corners=3pt,inner sep=4pt,fill=blue!50!black,text width= 0.95\columnwidth] {\color{white}\sectionlabel#1};
%\end{tikzpicture}
%  }
%\titlespacing*{\section}{0pt}{15pt}{10pt}
%
%
%% custom footer
%\usepackage{fancyhdr}
%\makeatletter
%\pagestyle{fancy}
%\fancyhead{}
%\fancyfoot[C]{\footnotesize \@author \ - \ies}
%\renewcommand{\headrulewidth}{0pt}
%\renewcommand{\footrulewidth}{0pt}
%\makeatother
%\usepackage{multirow} % para las tablas
%
%
%\title{Funciones}
%\author{Departamento de Matemáticas}
%\date{2014}
%\newcommand{\ies}{IES Pedro Cerrada}
%
%
%
%\begin{document}
%
%\maketitle
%
%
%
%\begin{multicols*}{2}


\section{Función, dominio y recorrido}

En el lenguaje matemático se dice que $y$ es \textbf{función} de $x$ cuando $y$ depende de $x$.

El conjunto de los posibles valores de la
variable independiente se llama \textbf{dominio de la función} ($Dom(f)$); y el conjunto de valores que toma la variable dependiente, \textbf{imagen o recorrido de la función} ($Im(f)$).

\subsection{Ejemplo}
$y=x^2$ o $f(x)=x^2$, $x$ es la variable independiente e $y$ es la variable dependiente. El $Dom(f)=\left\lbrace x \in \mathrm{R} | \exists y=f(x)\right\rbrace$ 	

\begin{tikzpicture}[domain=-3:3,>=triangle 45, scale=1]

\tikzmath{
			\a = 1/4; \b = 1; \c = -2; 
			\d = 3;
          }
          
\draw[color=red]    plot (\x,{\a*(\x)^2 + \b*\x + \c})             node[right] {$f(x) =\a x^2$}; 
\draw[very thin,color=gray] (-3.5,\c - 0.5) grid (3.5,\a * 3^2 + \b * 3 + \c - 0.1);
\draw[<->] (-3.5,0) -- (3.9,0) node[right] {$x$};
\draw[<->] (0,\c - 0.5) -- (0,\a * 3^2 + \b * 3 + \c) node[above] {$y$};

\end{tikzpicture}






%\end{multicols*}
%
%\end{document}
