%% Options for packages loaded elsewhere
%\PassOptionsToPackage{unicode=true}{hyperref}
%\PassOptionsToPackage{hyphens}{url}
%%
%\documentclass[
%]{article}
%\usepackage{lmodern}
%\usepackage{amssymb,amsmath}
%\usepackage{ifxetex,ifluatex}
%\ifnum 0\ifxetex 1\fi\ifluatex 1\fi=0 % if pdftex
%  \usepackage[T1]{fontenc}
%  \usepackage[utf8]{inputenc}
%  \usepackage{textcomp} % provides euro and other symbols
%\else % if luatex or xelatex
%  \usepackage{unicode-math}
%  \defaultfontfeatures{Scale=MatchLowercase}
%  \defaultfontfeatures[\rmfamily]{Ligatures=TeX,Scale=1}
%\fi
%% Use upquote if available, for straight quotes in verbatim environments
%\IfFileExists{upquote.sty}{\usepackage{upquote}}{}
%\IfFileExists{microtype.sty}{% use microtype if available
%  \usepackage[]{microtype}
%  \UseMicrotypeSet[protrusion]{basicmath} % disable protrusion for tt fonts
%}{}
%\makeatletter
%\@ifundefined{KOMAClassName}{% if non-KOMA class
%  \IfFileExists{parskip.sty}{%
%    \usepackage{parskip}
%  }{% else
%    \setlength{\parindent}{0pt}
%    \setlength{\parskip}{6pt plus 2pt minus 1pt}}
%}{% if KOMA class
%  \KOMAoptions{parskip=half}}
%\makeatother
%\usepackage{xcolor}
%\IfFileExists{xurl.sty}{\usepackage{xurl}}{} % add URL line breaks if available
%\IfFileExists{bookmark.sty}{\usepackage{bookmark}}{\usepackage{hyperref}}
%\hypersetup{
%  hidelinks,
%}
%\urlstyle{same} % disable monospaced font for URLs
%\setlength{\emergencystretch}{3em} % prevent overfull lines
%\providecommand{\tightlist}{%
%  \setlength{\itemsep}{0pt}\setlength{\parskip}{0pt}}
%\setcounter{secnumdepth}{-\maxdimen} % remove section numbering
%% Redefines (sub)paragraphs to behave more like sections
%\ifx\paragraph\undefined\else
%  \let\oldparagraph\paragraph
%  \renewcommand{\paragraph}[1]{\oldparagraph{#1}\mbox{}}
%\fi
%\ifx\subparagraph\undefined\else
%  \let\oldsubparagraph\subparagraph
%  \renewcommand{\subparagraph}[1]{\oldsubparagraph{#1}\mbox{}}
%\fi
%
%% Set default figure placement to htbp
%\makeatletter
%\def\fps@figure{htbp}
%\makeatother
%
%
%\date{}
%
%\begin{document}
%

\definecolor{wrwrwr}{rgb}{0.3803921568627451,0.3803921568627451,0.3803921568627451}
\definecolor{rvwvcq}{rgb}{0.08235294117647059,0.396078431372549,0.7529411764705882}

\hypertarget{rectas-y-puntos-notables-de-un-triuxe1ngulo}{%
\section{Rectas y puntos notables de un
triángulo}\label{rectas-y-puntos-notables-de-un-triuxe1ngulo}}

\hypertarget{mediana}{%
\subsection{Mediana}\label{mediana}}

La \textbf{mediana} es el segmento que va del punto medio de un lado al
vértice opuesto.

Al punto dónde se cortan las medianas de un triángulo se le llama
\textbf{baricentro} y constituye el centro de gravedad del polígono.

%
%\documentclass[10pt]{standalone}
%\usepackage{pgf,tikz,pgfplots}
%\pgfplotsset{compat=1.15}
%\usepackage{mathrsfs}
%\usetikzlibrary{arrows}
%%\pagestyle{empty}
%
%\begin{document}
%% Medianas y baricentro
%
%\definecolor{wrwrwr}{rgb}{0.3803921568627451,0.3803921568627451,0.3803921568627451}
%\definecolor{rvwvcq}{rgb}{0.08235294117647059,0.396078431372549,0.7529411764705882}
\begin{tikzpicture}[line cap=round,line join=round,>=triangle 45,x=1cm,y=1cm]
\clip(-6.154568000000002,-1.8982211062725844) rectangle (4.106232000000003,7.1694626146576494);
\fill[line width=2pt,color=rvwvcq,fill=rvwvcq,fill opacity=0.10000000149011612] (-5,-1) -- (3,-1) -- (0,6) -- cycle;
\draw [line width=2pt,color=rvwvcq] (-5,-1)-- (3,-1);
\draw [line width=2pt,color=rvwvcq] (3,-1)-- (0,6);
\draw [line width=2pt,color=rvwvcq] (0,6)-- (-5,-1);
\draw [line width=2pt,color=wrwrwr] (-2.5,2.5)-- (3,-1);
\draw [line width=2pt,color=wrwrwr] (-1,-1)-- (0,6);
\draw [line width=2pt,color=wrwrwr] (-5,-1)-- (1.5,2.5);
\begin{scriptsize}
\draw [fill=rvwvcq] (-5,-1) circle (2.5pt);
\draw[color=rvwvcq] (-4.911871111111114,-0.7361607091015785) node {$A$};
\draw [fill=rvwvcq] (3,-1) circle (2.5pt);
\draw[color=rvwvcq] (3.091552888888891,-0.7361607091015785) node {$B$};
\draw [fill=rvwvcq] (0,6) circle (2.5pt);
\draw[color=rvwvcq] (0.09311911111111175,6.260285827233702) node {$C$};
\draw [fill=wrwrwr] (-1,-1) circle (2pt);
\draw[color=wrwrwr] (-0.9101591111111109,-0.7602448624108221) node {$D$};
\draw [fill=wrwrwr] (-2.5,2.5) circle (2pt);
\draw[color=wrwrwr] (-2.403675555555556,2.731957367429507) node {$E$};
\draw [fill=wrwrwr] (1.5,2.5) circle (2pt);
\draw[color=wrwrwr] (1.586635555555557,2.731957367429507) node {$F$};
\draw [fill=wrwrwr] (-0.6666666666666666,1.3333333333333333) circle (2pt);
\draw[color=wrwrwr] (0.6843262222222229,1.359160628802619) node {$Baricentro$};
\end{scriptsize}
\end{tikzpicture}
%\end{document}

\hypertarget{mediatriz}{%
\subsection{Mediatriz}\label{mediatriz}}

La \textbf{mediatriz} de un segmento es la recta perpendicular al punto medio.
Geométricamente son los puntos del plano que equidistan a ambos extremos
del segmento.

En un triángulo llamaremos mediatriz a la mediatriz de cada uno de los
lados.

El punto de corte de las tres mediatrices equidista a los tres vértices
del triángulo. Ha dicho punto se le llama \textbf{circuncentro} porque permite
circunscribir el triángulo en una circunferencia de centro dicho punto.

%\documentclass[10pt]{article}
%\usepackage{pgf,tikz,pgfplots}
%\pgfplotsset{compat=1.15}
%\usepackage{mathrsfs}
%\usetikzlibrary{arrows}
%\pagestyle{empty}
%\begin{document}
%\definecolor{wrwrwr}{rgb}{0.3803921568627451,0.3803921568627451,0.3803921568627451}
%\definecolor{rvwvcq}{rgb}{0.08235294117647059,0.396078431372549,0.7529411764705882}
\begin{tikzpicture}[grow=right, sloped, scale=0.7]\clip(-6.154568000000002,-3.8982211062725844) rectangle (4.2,7);
\fill[line width=2pt,color=rvwvcq,fill=rvwvcq,fill opacity=0.10000000149011612] (-5,-1) -- (3,-1) -- (0,6) -- cycle;
\draw [line width=2pt,color=rvwvcq] (-5,-1)-- (3,-1);
\draw [line width=2pt,color=rvwvcq] (3,-1)-- (0,6);
\draw [line width=2pt,color=rvwvcq] (0,6)-- (-5,-1);
\draw [line width=2pt,color=wrwrwr] (-1,-6.399108752051755) -- (-1,9.599938859846416);
\draw [line width=2pt,color=wrwrwr,domain=-13.636070873907869:13.273820660511266] plot(\x,{(--5-5*\x)/7});
\draw [line width=2pt,color=wrwrwr,domain=-13.636070873907869:13.273820660511266] plot(\x,{(-13-3*\x)/-7});
\draw [line width=2pt,color=wrwrwr] (-1,1.4285714285714286) circle (4.679525529759771cm);
\begin{scriptsize}
\draw [fill=rvwvcq] (-5,-1) circle (2.5pt);
\draw[color=rvwvcq] (-4.808111733768611,-0.46196217732391776) node {$A$};
\draw [fill=rvwvcq] (3,-1) circle (2.5pt);
\draw[color=rvwvcq] (3.191486629145785,-0.46196217732391776) node {$B$};
\draw [fill=rvwvcq] (0,6) circle (2.5pt);
\draw[color=rvwvcq] (0.18572038035842364,6.537621152881533) node {$C$};
\draw [fill=wrwrwr] (-1,-1) circle (2pt);
\draw[color=wrwrwr] (-0.8083125523114124,-0.5119592011110996) node {$D$};
\draw [fill=wrwrwr] (-2.5,2.5) circle (2pt);
\draw[color=wrwrwr] (-2.2993619513161665,2.9878324639916256) node {$E$};
\draw [fill=wrwrwr] (1.5,2.5) circle (2pt);
\draw[color=wrwrwr] (1.700437230141031,2.9878324639916256) node {$F$};
\draw [fill=wrwrwr] (-1,1.4285714285714286) circle (2pt);
\draw[color=wrwrwr] (1.1797533130282598,1.3129321671210357) node {$Circuncentro$};
\end{scriptsize}
\end{tikzpicture}
%\end{document}

\hypertarget{altura}{%
\subsection{Altura}\label{altura}}

Llamaremos \textbf{altura} de un triángulo al segmento perpendicular a un lado y
pasa por el vértice opuesto.

Al punto de corte de las alturas se le llama \textbf{ortocentro}.

%\documentclass[10pt]{article}
%\usepackage{pgf,tikz,pgfplots}
%%\pgfplotsset{compat=1.15}
%\usepackage{mathrsfs}
%\usetikzlibrary{arrows}
%\pagestyle{empty}
%\newcommand{\degre}{\ensuremath{^\circ}}
%\begin{document}
%\definecolor{wrwrwr}{rgb}{0.3803921568627451,0.3803921568627451,0.3803921568627451}
%\definecolor{rvwvcq}{rgb}{0.98235294117647059,0.396078431372549,0.7529411764705882}
\begin{tikzpicture}[grow=right, sloped, scale=0.7]\clip(-6.154568000000002,-3.8982211062725844) rectangle (4.2,7);
\clip(-11.145015598814782,-9.076966577456616) rectangle (18.28831648795786,8.422371196936597);
\fill[line width=2pt,color=rvwvcq,fill=rvwvcq,fill opacity=0.10000000149011612] (-5,-1) -- (3,-1) -- (0,6) -- cycle;
\draw[line width=1.6pt,fill=black,fill opacity=0.10000000149011612] (-2.616480014189626,2.336927980134524) -- (-2.1696242105403662,2.0177452632421953) -- (-1.8504414936480373,2.464601066891455) -- (-2.2972972972972974,2.7837837837837838) -- cycle; 
\draw[line width=1.6pt,fill=black,fill opacity=0.10000000149011612] (1.5423027784864105,2.401293516865042) -- (1.0375609857592993,2.1849756056962804) -- (1.2538788969280612,1.6802338129691692) -- (1.7586206896551724,1.896551724137931) -- cycle; 
\draw[line width=1.6pt,fill=black,fill opacity=0.10000000149011612] (0.5491427100652383,-1) -- (0.5491427100652384,-0.4508572899347618) -- (0,-0.4508572899347617) -- (0,-1) -- cycle; 
\draw [line width=2pt,color=rvwvcq] (-5,-1)-- (3,-1);
\draw [line width=2pt,color=rvwvcq] (3,-1)-- (0,6);
\draw [line width=2pt,color=rvwvcq] (0,6)-- (-5,-1);
\draw [line width=2pt,color=wrwrwr] (0,-9.076966577456616) -- (0,8.422371196936597);
\draw [line width=2pt,color=wrwrwr,domain=-11.145015598814782:18.28831648795786] plot(\x,{(-8-3*\x)/-7});
\draw [line width=2pt,color=wrwrwr,domain=-11.145015598814782:18.28831648795786] plot(\x,{(--8-5*\x)/7});
\begin{scriptsize}
\draw [fill=rvwvcq] (-5,-1) circle (2.5pt);
\draw[color=rvwvcq] (-4.802740874752074,-0.42299719371372185) node {$A$};
\draw [fill=rvwvcq] (3,-1) circle (2.5pt);
\draw[color=rvwvcq] (3.1962913282494623,-0.42299719371372185) node {$B$};
\draw [fill=rvwvcq] (0,6) circle (2.5pt);
\draw[color=rvwvcq] (0.21930523328125284,6.576737916043563) node {$C$};
\draw[color=black] (-1.0750365471397076,2.530016055715133) node {$\alpha = 90\textrm{\degre}$};
\draw[color=black] (2.3937994243884666,2.256588902990239) node {$\beta = 90\textrm{\degre}$};
\draw[color=black] (1.4877601780937941,-0.4776826242587007) node {$\gamma = 90\textrm{\degre}$};
\draw [fill=wrwrwr] (0,1.1428571428571428) circle (2pt);
\draw[color=wrwrwr] (1.8501758766116632,1.1355375768181737) node {$Ortocentro$};
\end{scriptsize}
\end{tikzpicture}
%\end{document}

\hypertarget{bisectriz}{%
\subsection{Bisectriz}\label{bisectriz}}

Llamamos \textbf{bisectriz} de un ángulo a la semirrecta que divide al ángulo en
dos ángulos iguales.
En un triángulo tendremos las tres bisectrices correspondientes a cada uno de los tres ángulos.

Llamaremos \textbf{incentro} al punto de corte de las bisectrices de un triángulo.

%\documentclass[10pt]{article}
%\usepackage{pgf,tikz,pgfplots}
%\pgfplotsset{compat=1.15}
%\usepackage{mathrsfs}
%\usetikzlibrary{arrows}
%\pagestyle{empty}
%\newcommand{\degre}{\ensuremath{^\circ}}
%\begin{document}
%\definecolor{wrwrwr}{rgb}{0.3803921568627451,0.3803921568627451,0.3803921568627451}
%\definecolor{rvwvcq}{rgb}{0.08235294117647059,0.396078431372549,0.7529411764705882}
\begin{tikzpicture}[grow=right, sloped, scale=0.7]\clip(-6.154568000000002,-3.8982211062725844) rectangle (4.2,7);
\fill[line width=2pt,color=rvwvcq,fill=rvwvcq,fill opacity=0.10000000149011612] (-5,-1) -- (3,-1) -- (0,6) -- cycle;
\draw [shift={(0,6)},line width=1.6pt,fill=black,fill opacity=0.10000000149011612] (0,0) -- (-125.53767779197437:0.7766050682525762) arc (-125.53767779197437:-66.8014094863518:0.7766050682525762) -- cycle;
\draw [shift={(-5,-1)},line width=1.6pt,fill=black,fill opacity=0.10000000149011612] (0,0) -- (0:0.7766050682525762) arc (0:54.462322208025626:0.7766050682525762) -- cycle;
\draw [shift={(3,-1)},line width=1.6pt,fill=black,fill opacity=0.10000000149011612] (0,0) -- (113.19859051364813:0.7766050682525762) arc (113.19859051364813:180:0.7766050682525762) -- cycle;
\draw [shift={(3,-1)},line width=1.6pt,fill=black,fill opacity=0.10000000149011612] (0,0) -- (113.19859051364813:0.7766050682525762) arc (113.19859051364813:146.59929525682406:0.7766050682525762) -- cycle;
\draw [shift={(0,6)},line width=1.6pt,fill=black,fill opacity=0.10000000149011612] (0,0) -- (-125.53767779197437:0.7766050682525762) arc (-125.53767779197437:-96.1695436391631:0.7766050682525762) -- cycle;
\draw [shift={(-5,-1)},line width=1.6pt,fill=black,fill opacity=0.10000000149011612] (0,0) -- (0:0.7766050682525762) arc (0:27.231161104012813:0.7766050682525762) -- cycle;
\draw [line width=2pt,color=rvwvcq] (-5,-1)-- (3,-1);
\draw [line width=2pt,color=rvwvcq] (3,-1)-- (0,6);
\draw [line width=2pt,color=rvwvcq] (0,6)-- (-5,-1);
\draw [line width=2pt,color=wrwrwr,domain=-13.604264981614604:15.829067105158035] plot(\x,{(-0.8166319340003523--0.5504910087462067*\x)/-0.8348410922382679});
\draw [line width=2pt,color=wrwrwr,domain=-13.604264981614604:15.829067105158035] plot(\x,{(-0.6448253150018926-0.9942082320611364*\x)/-0.10747088583364875});
\draw [line width=2pt,color=wrwrwr,domain=-13.604264981614604:15.829067105158035] plot(\x,{(--1.3987402615700915--0.4575815808579403*\x)/0.88916764271961});
\draw [line width=2pt,color=wrwrwr] (-0.5067239194106411,1.3123202795579019) circle (2.313929526200274cm);
\begin{scriptsize}
\draw [fill=rvwvcq] (-5,-1) circle (2.5pt);
\draw[color=rvwvcq] (-4.8027408747520735,-0.42299719371372047) node {$A$};
\draw [fill=rvwvcq] (3,-1) circle (2.5pt);
\draw[color=rvwvcq] (3.196291328249462,-0.42299719371372047) node {$B$};
\draw [fill=rvwvcq] (0,6) circle (2.5pt);
\draw[color=rvwvcq] (0.2193052332812528,6.576737916043568) node {$C$};
\draw[color=black] (-0.7643945198386772,5.838484603686354) node {$\alpha = 58.7\textrm{\degre}$};
\draw[color=black] (-2.498812505602764,0.2332279728260253) node {$\beta = 54.5\textrm{\degre}$};
\draw[color=black] (3.222178163857881,0.5066551255509193) node {$\gamma = 66.8\textrm{\degre}$};
\draw[color=black] (2.005496890262178,-0.5050253395311887) node {$\delta = 33.4\textrm{\degre}$};
\draw[color=black] (1.6689680273527285,5.07288857605665) node {$\epsilon = 29.4\textrm{\degre}$};
\draw[color=black] (-2.265830985126991,-0.47768262425869923) node {$\zeta = 27.2\textrm{\degre}$};
\draw [fill=wrwrwr] (-0.5067239194106411,1.3123202795579019) circle (2pt);
\draw[color=wrwrwr] (-0.5572998349713234,0.42462697973345115) node {$Incentro$};
\draw [fill=wrwrwr] (-2.438487372426055,2.5861176786035225) circle (2pt);
\draw[color=wrwrwr] (-2.239944149518572,3.131555791709902) node {$D$};
\end{scriptsize}
\end{tikzpicture}
%\end{document}

%\end{document}
